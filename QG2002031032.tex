%%%%%%%%%%%%%%%%%%%%%%%%%%%%%%%%%%%%%%%%%%%%%%%%%%%%%%%%%%%%%%%%%%%%%%%%%%%%%%%%%%%%%%%%%%%%%%%%%%%%%%%%%%%%%%%%%%%%%%%%%%
\documentclass[11pt]{article}
%%%%%%%%%%%%%%%%%%%%%%%%%%%%%%%%%%%%%%%%%%%%%%%%%%%%%%%%%%%%%%%%%%%%%%%%%%%%%%%%%%%%%%%%%%%%%%%%%%%%%%%%%%%%%%%%%%%%%%%%%%
\textheight 22.5truecm \textwidth 15truecm
\setlength{\oddsidemargin}{-0.00in}
\setlength{\topmargin}{-.5cm}
%%%%%%%%%%%%%%%%%%%%%%%%%%%%%%%%%%%%%%%%%%%%%%%%%%%%%%%%%%%%%%%%%%%%%%%%%%%%%%%%%%%%%%%%%%%%%%%%%%%%%%%%%%%%%%%%%%%%%%%%%%
\usepackage{authblk}
\usepackage{amssymb}
\usepackage[all]{xy}
\usepackage{mathrsfs}
\usepackage{amsmath,amssymb,amsthm}
\usepackage{extarrows}
%\usepackage[mathscr]{eucal}
\usepackage{mathrsfs}
%%%%%%%%%%%%%%%%%%%%%%%%%%%%%%%%%%%%%%%%%%%%%%%%%%%%%%%%%%%%%%%%%%%%%%%%%%%%%%%%%%%%%%%%%%%%%%%%%%%%%%%%%%%%%%%%%%%%%%%%%%
\newtheorem{theorem}{Theorem}[section]
\newtheorem{lemma}[theorem]{Lemma}
\newtheorem{proposition}[theorem]{Proposition}
\newtheorem{definition}[theorem]{Definition}
\newtheorem{question}[theorem]{Question}
\newtheorem{remark}[theorem]{Remark}
\newtheorem{corollary}[theorem]{Corollary}
\newtheorem{example}[theorem]{Example}
\newtheorem{problem}[theorem]{Problem}
%%%%%%%%%%%%%%%%%%%%%%%%%%%%%%%%%%%%%%%%%%%%%%%%%%%%%%%%%%%%%%%%%%%%%%%%%%%%%%%%%%%%%%%%%%%%%%%%%%%%%%%%%%%%%%%%%%%%%%%%%%
\def\to{\rightarrow}
\def\f{\mathfrak}
\def\m{\mathbb}
\def\c{\mathcal}
\def\b{\mathbf}
\def\r{\mathrm}
\def\bb{\mathbb}
\def\s{\mathscr}
%%%%%%%%%%%%%%%%%%%%%%%%%%%%%%%%%%%%%%%%%%%%%%%%%%%%%%%%%%%%%%%%%%%%%%%%%%%%%%%%%%%%%%%%%%%%%%%%%%%%%%%%%%%%%%%%%%%%%%%%%%
\date{}
%%%%%%%%%%%%%%%%%%%%%%%%%%%%%%%%%%%%%%%%%%%%%%%%%%%%%%%%%%%%%%%%%%%%%%%%%%%%%%%%%%%%%%%%%%%%%%%%%%%%%%%%%%%%%%%%%%%%%%%%%%
\begin{document}
%\ttfamily
%\sffamily
%\bfseries
%%%%%%%%%%%%%%%%%%%%%%%%%%%%%%%%%%%%%%%%%%%%%%%%%%%%%%%%%%%%%%%%%%%%%%%%%%%%%%%%%%%%%%%%%%%%%%%%%%%%%%%%%%%%%%%%%%%%%%%%%%
\title{Harmonic Analysis on Quantum Groups}
%%%%%%%%%%%%%%%%%%%%%%%%%%%%%%%%%%%%%%%%%%%%%%%%%%%%%%%%%%%%%%%%%%%%%%%%%%%%%%%%%%%%%%%%%%%%%%%%%%%%%%%%%%%%%%%%%%%%%%%%%%
\author{Maysam Maysami Sadr
\thanks{sadr@iasbs.ac.ir}
}
\author{Danial Bouzarjomehri Amnieh \thanks{danial.bouzarj@iasbs.ac.ir}}
\affil{Department of Mathematics, Institute for Advanced Studies in Basic Sciences, Zanjan, Iran}
%%%%%%%%%%%%%%%%%%%%%%%%%%%%%%%%%%%%%%%%%%%%%%%%%%%%%%%%%%%%%%%%%%%%%%%%%%%%%%%%%%%%%%%%%%%%%%%%%%%%%%%%%%%%%%%%%%%%%%%%%%
\maketitle
%%%%%%%%%%%%%%%%%%%%%%%%%%%%%%%%%%%%%%%%%%%%%%%%%%%%%%%%%%%%%%%%%%%%%%%%%%%%%%%%%%%%%%%%%%%%%%%%%%%%%%%%%%%%%%%%%%%%%%%%%%
%\begin{abstract}
%\textbf{MSC 2010.}
%\textbf{Keywords.}
%\end{abstract}
%%%%%%%%%%%%%%%%%%%%%%%%%%%%%%%%%%%%%%%%%%%%%%%%%%%%%%%%%%%%%%%%%%%%%%%%%%%%%%%%%%%%%%%%%%%%%%%%%%%%%%%%%%%%%%%%%%%%%%%%%%
%%%%%%%%%%%%%%%%%%%%%%%%%%%%%%%%%%%%%%%%%%%%%%%%%%%%%%%%%%%%%%%%%%%%%%%%%%%%%%%%%%%%%%%%%%%%%%%%%%%%%%%%%%%%%%%%%%%%%%%%%%
%%%%%%%%%%%%%%%%%%%%%%%%%%%%%%%%%%%%%%%%%%%%%%%%%%%%%%%%%%%%%%%%%%%%%%%%%%%%%%%%%%%%%%%%%%%%%%%%%%%%%%%%%%%%%%%%%%%%%%%%%%
%%%%%%%%%%%%%%%%%%%%%%%%%%%%%%%%%%%%%%%%%%%%%%%%%%%%%%%%%%%%%%%%%%%%%%%%%%%%%%%%%%%%%%%%%%%%%%%%%%%%%%%%%%%%%%%%%%%%%%%%%%
%%%%%%%%%%%%%%%%%%%%%%%%%%%%%%%%%%%%%%%%%%%%%%%%%%%%%%%%%%%%%%%%%%%%%%%%%%%%%%%%%%%%%%%%%%%%%%%%%%%%%%%%%%%%%%%%%%%%%%%%%%
\section{General Information and Explanations}
In literatures the phrase `quantum group' refer to an algebra (over a ring) with a type of comultiplication which satisfies some conditions
analogues to the properties of a Hopf algebra. Of course many authors call Hopf algebras also quantum groups. Also, some authors
refers to such objects as `quantum groups' if the underlying algebra is over the field of complex numbers, the algebra is involutive, and
the algebra is not commutative and is not cocommutative.
In this section we address some general information, references, and terminology of the very vast theory of quantum groups.
See also \cite{Van-Daele8} for some similar discussions.
%%%%%%%%%%%%%%%%%%%%%%%%%%%%%%%%%%%%%%%%%%%%%%%%%%%%%%%%%%%%%%%%%%%%%%%%%%%%%%%%%%%%%%%%%%%%%%%%%%%%%%%%%%%%%%%%%%%%%%%%%%
%%%%%%%%%%%%%%%%%%%%%%%%%%%%%%%%%%%%%%%%%%%%%%%%%%%%%%%%%%%%%%%%%%%%%%%%%%%%%%%%%%%%%%%%%%%%%%%%%%%%%%%%%%%%%%%%%%%%%%%%%%
\subsection{References for Some Preliminaries}
Besides some general information in algebra, C*-algebra, and harmonic analysis, all the other preliminaries that we will need are given in
\cite{Van-Daele7}, appendixes of \cite{MasudaNakagamiWoronowicz1}, and the appendix of \cite{Van-Daele2}.
See for Sweedler notation see \cite[Section 2]{Van-Daele7}. For a good source for Tomita-Takesaki modular theory see \cite{Lledo1}.
See \cite[Example 2.9]{Lledo1} for the relation between modular functions for locally compact groups and modular theory.
For basic von Neumann algebra theory \cite{Uijlen1} is also a good source.
%%%%%%%%%%%%%%%%%%%%%%%%%%%%%%%%%%%%%%%%%%%%%%%%%%%%%%%%%%%%%%%%%%%%%%%%%%%%%%%%%%%%%%%%%%%%%%%%%%%%%%%%%%%%%%%%%%%%%%%%%%
%%%%%%%%%%%%%%%%%%%%%%%%%%%%%%%%%%%%%%%%%%%%%%%%%%%%%%%%%%%%%%%%%%%%%%%%%%%%%%%%%%%%%%%%%%%%%%%%%%%%%%%%%%%%%%%%%%%%%%%%%%
\subsection{Locally Compact Groups}
A good reference is \cite{DeitmarEchterhoff1}. By a locally compact group we mean a group $G$ together with a locally compact Hausdorff
topology on $G$ such that the structural group mappings $G\times G\to G, G\to G$, given by $(g,h)\mapsto gh$ and $g\mapsto g^{-1}$
are continuous. Let $G$ be a locally compact group. There exists a Borel regular measure $\lambda$ on $G$ such that is left invariant
(i.e. $\lambda(gB)=\lambda(B)$ for every $g\in G$ and any Borel subset $B\subseteq G$) and $\r{supp}(\lambda)=G$ (i.e. $\lambda$ is
nonzero on any nonempty open subset of $G$). Such a measure $\lambda$ is called a left Haar measure. The right Haar measures
are defined similarly. If $\lambda,\lambda'$ are left Haar measures on $G$ then there is a strictly positive real $r$ such that $\lambda'=r\lambda$.
The similar fact is satisfied for right Haar measures. If $\lambda$ is a left Haar measure then $\rho$ defined by $\rho(B)=\lambda(B^{-1})$
for Borel subsets of $G$, is a right Haar measure. $G$ is called unimodular if every left Haar measure is also a right Haar measure.
Any abelian or compact group is unimodular.\\
Let $\lambda$ be a left Haar measure on $G$. Then for every $g\in G$, the assignment
$B\mapsto\lambda(Bg)$ defines another left Haar measure $\lambda_g$, and thus there is a positive real $\Delta(g)$ such that
$\lambda_g=\Delta(g)\lambda$. It is not hard to see that $\Delta$ is a continuous group homomorphism from $G$ to the multiplicative group
$\bb{R}^+$. It is clear that $G$ is unimodular if and only if $\Delta=1$.\\
Let $G$ be an abelian locally compact group. Let $\widehat{G}$ denote the set of all continuous
group homomorphisms from $G$ to the circle $\bb{T}$. Then $\widehat{G}$ with the pointwise multiplication and compact-open topology
is an abelian locally compact group. $\widehat{G}$ is called Pontryagin dual of $G$. The canonical mapping $G\to\widehat{\widehat{G}}$
is an isomorphism of locally compact groups. The dual of an abelian compact group is an abelian discrete group and the dual of an abelian
discrete group is an abelian compact group.
%%%%%%%%%%%%%%%%%%%%%%%%%%%%%%%%%%%%%%%%%%%%%%%%%%%%%%%%%%%%%%%%%%%%%%%%%%%%%%%%%%%%%%%%%%%%%%%%%%%%%%%%%%%%%%%%%%%%%%%%%%
%%%%%%%%%%%%%%%%%%%%%%%%%%%%%%%%%%%%%%%%%%%%%%%%%%%%%%%%%%%%%%%%%%%%%%%%%%%%%%%%%%%%%%%%%%%%%%%%%%%%%%%%%%%%%%%%%%%%%%%%%%
\subsection{Groupoids}




%%%%%%%%%%%%%%%%%%%%%%%%%%%%%%%%%%%%%%%%%%%%%%%%%%%%%%%%%%%%%%%%%%%%%%%%%%%%%%%%%%%%%%%%%%%%%%%%%%%%%%%%%%%%%%%%%%%%%%%%%%
%%%%%%%%%%%%%%%%%%%%%%%%%%%%%%%%%%%%%%%%%%%%%%%%%%%%%%%%%%%%%%%%%%%%%%%%%%%%%%%%%%%%%%%%%%%%%%%%%%%%%%%%%%%%%%%%%%%%%%%%%%
\subsection{Locally Compact Groupoids}


%%%%%%%%%%%%%%%%%%%%%%%%%%%%%%%%%%%%%%%%%%%%%%%%%%%%%%%%%%%%%%%%%%%%%%%%%%%%%%%%%%%%%%%%%%%%%%%%%%%%%%%%%%%%%%%%%%%%%%%%%%
%%%%%%%%%%%%%%%%%%%%%%%%%%%%%%%%%%%%%%%%%%%%%%%%%%%%%%%%%%%%%%%%%%%%%%%%%%%%%%%%%%%%%%%%%%%%%%%%%%%%%%%%%%%%%%%%%%%%%%%%%%
\subsection{Hopf Algebras}

%%%%%%%%%%%%%%%%%%%%%%%%%%%%%%%%%%%%%%%%%%%%%%%%%%%%%%%%%%%%%%%%%%%%%%%%%%%%%%%%%%%%%%%%%%%%%%%%%%%%%%%%%%%%%%%%%%%%%%%%%%
%%%%%%%%%%%%%%%%%%%%%%%%%%%%%%%%%%%%%%%%%%%%%%%%%%%%%%%%%%%%%%%%%%%%%%%%%%%%%%%%%%%%%%%%%%%%%%%%%%%%%%%%%%%%%%%%%%%%%%%%%%
\subsection{C*-Hopf Algebras}



%%%%%%%%%%%%%%%%%%%%%%%%%%%%%%%%%%%%%%%%%%%%%%%%%%%%%%%%%%%%%%%%%%%%%%%%%%%%%%%%%%%%%%%%%%%%%%%%%%%%%%%%%%%%%%%%%%%%%%%%%%
%%%%%%%%%%%%%%%%%%%%%%%%%%%%%%%%%%%%%%%%%%%%%%%%%%%%%%%%%%%%%%%%%%%%%%%%%%%%%%%%%%%%%%%%%%%%%%%%%%%%%%%%%%%%%%%%%%%%%%%%%%
\subsection{Co-Involutive Hopf von Neumann Algebras\\(von Neumann Algebraic Setting, Ernest)}
The best reference with all preliminaries is \cite[Chapter 1]{EnockSchwartz1}.



%%%%%%%%%%%%%%%%%%%%%%%%%%%%%%%%%%%%%%%%%%%%%%%%%%%%%%%%%%%%%%%%%%%%%%%%%%%%%%%%%%%%%%%%%%%%%%%%%%%%%%%%%%%%%%%%%%%%%%%%%%
%%%%%%%%%%%%%%%%%%%%%%%%%%%%%%%%%%%%%%%%%%%%%%%%%%%%%%%%%%%%%%%%%%%%%%%%%%%%%%%%%%%%%%%%%%%%%%%%%%%%%%%%%%%%%%%%%%%%%%%%%%
\subsection{Kac Algebras\\(von Neumann Algebraic Setting, Enock-Schwartz)}
The best reference with all preliminaries is \cite[Chapter 2]{EnockSchwartz1}.



%%%%%%%%%%%%%%%%%%%%%%%%%%%%%%%%%%%%%%%%%%%%%%%%%%%%%%%%%%%%%%%%%%%%%%%%%%%%%%%%%%%%%%%%%%%%%%%%%%%%%%%%%%%%%%%%%%%%%%%%%%
%%%%%%%%%%%%%%%%%%%%%%%%%%%%%%%%%%%%%%%%%%%%%%%%%%%%%%%%%%%%%%%%%%%%%%%%%%%%%%%%%%%%%%%%%%%%%%%%%%%%%%%%%%%%%%%%%%%%%%%%%%
\subsection{Compact Quantum Groups\\(C*-Algebraic Setting, Woronowicz)}\label{2001301307}
The main paper is \cite{Woronowicz1}. The basic examples are given in \cite{Woronowicz2}.
See also \cite{MaesVan-Daele1} or \cite[Chapter 1]{NeshveyevTuset1} for more details.
See \cite[Proposition 3.2]{MaesVan-Daele1} for the proof of the fact that any compact semigroup
with left and right cancelation properties is a group. See \cite[Section 8]{MaesVan-Daele1} for the proof that the dual of any compact
quantum group is a discrete quantum group in the sense of Subsection \ref{2001281638}.

%%%%%%%%%%%%%%%%%%%%%%%%%%%%%%%%%%%%%%%%%%%%%%%%%%%%%%%%%%%%%%%%%%%%%%%%%%%%%%%%%%%%%%%%%%%%%%%%%%%%%%%%%%%%%%%%%%%%%%%%%%
%%%%%%%%%%%%%%%%%%%%%%%%%%%%%%%%%%%%%%%%%%%%%%%%%%%%%%%%%%%%%%%%%%%%%%%%%%%%%%%%%%%%%%%%%%%%%%%%%%%%%%%%%%%%%%%%%%%%%%%%%%
\subsection{Locally Compact Quantum Groups\\(Reduced, C*-Algebraic Setting, Kustermans-Vaes}
\cite{KustermansVaes1}


%%%%%%%%%%%%%%%%%%%%%%%%%%%%%%%%%%%%%%%%%%%%%%%%%%%%%%%%%%%%%%%%%%%%%%%%%%%%%%%%%%%%%%%%%%%%%%%%%%%%%%%%%%%%%%%%%%%%%%%%%%
%%%%%%%%%%%%%%%%%%%%%%%%%%%%%%%%%%%%%%%%%%%%%%%%%%%%%%%%%%%%%%%%%%%%%%%%%%%%%%%%%%%%%%%%%%%%%%%%%%%%%%%%%%%%%%%%%%%%%%%%%%
\subsection{Locally Compact Quantum Groups\\(von Neumann Algebraic Setting, Kustermans-Vaes}
\cite{KustermansVaes2} See also \cite{Van-Daele5} and \cite{Van-Daele4}.



%%%%%%%%%%%%%%%%%%%%%%%%%%%%%%%%%%%%%%%%%%%%%%%%%%%%%%%%%%%%%%%%%%%%%%%%%%%%%%%%%%%%%%%%%%%%%%%%%%%%%%%%%%%%%%%%%%%%%%%%%%
%%%%%%%%%%%%%%%%%%%%%%%%%%%%%%%%%%%%%%%%%%%%%%%%%%%%%%%%%%%%%%%%%%%%%%%%%%%%%%%%%%%%%%%%%%%%%%%%%%%%%%%%%%%%%%%%%%%%%%%%%%
\subsection{Locally Compact Quantum Groups\\(Universal, C*-Algebraic Setting, Kustermans}
\cite{Kustermans1}



%%%%%%%%%%%%%%%%%%%%%%%%%%%%%%%%%%%%%%%%%%%%%%%%%%%%%%%%%%%%%%%%%%%%%%%%%%%%%%%%%%%%%%%%%%%%%%%%%%%%%%%%%%%%%%%%%%%%%%%%%%
%%%%%%%%%%%%%%%%%%%%%%%%%%%%%%%%%%%%%%%%%%%%%%%%%%%%%%%%%%%%%%%%%%%%%%%%%%%%%%%%%%%%%%%%%%%%%%%%%%%%%%%%%%%%%%%%%%%%%%%%%%
\subsection{Weighted Hopf C*-algebras\\(C*-Algebraic Setting, Masuda-Nakagami-Woronowicz}
\cite{MasudaNakagamiWoronowicz1}
Closed with respect to duality and second dual is identity



%%%%%%%%%%%%%%%%%%%%%%%%%%%%%%%%%%%%%%%%%%%%%%%%%%%%%%%%%%%%%%%%%%%%%%%%%%%%%%%%%%%%%%%%%%%%%%%%%%%%%%%%%%%%%%%%%%%%%%%%%%
%%%%%%%%%%%%%%%%%%%%%%%%%%%%%%%%%%%%%%%%%%%%%%%%%%%%%%%%%%%%%%%%%%%%%%%%%%%%%%%%%%%%%%%%%%%%%%%%%%%%%%%%%%%%%%%%%%%%%%%%%%
\subsection{Discrete Quantum Groups\\(Purely Algebraic and Matrix Setting, Van Daele)}\label{2001281638}
The main paper is \cite{Van-Daele6}.

%%%%%%%%%%%%%%%%%%%%%%%%%%%%%%%%%%%%%%%%%%%%%%%%%%%%%%%%%%%%%%%%%%%%%%%%%%%%%%%%%%%%%%%%%%%%%%%%%%%%%%%%%%%%%%%%%%%%%%%%%%
%%%%%%%%%%%%%%%%%%%%%%%%%%%%%%%%%%%%%%%%%%%%%%%%%%%%%%%%%%%%%%%%%%%%%%%%%%%%%%%%%%%%%%%%%%%%%%%%%%%%%%%%%%%%%%%%%%%%%%%%%%
\subsection{Algebraic Quantum Hypergroups\\(Purely Algebraic Setting, Delvaux-Van Daele)}
The main paper is \cite{DelvauxVan-Daele1}.

%%%%%%%%%%%%%%%%%%%%%%%%%%%%%%%%%%%%%%%%%%%%%%%%%%%%%%%%%%%%%%%%%%%%%%%%%%%%%%%%%%%%%%%%%%%%%%%%%%%%%%%%%%%%%%%%%%%%%%%%%%
%%%%%%%%%%%%%%%%%%%%%%%%%%%%%%%%%%%%%%%%%%%%%%%%%%%%%%%%%%%%%%%%%%%%%%%%%%%%%%%%%%%%%%%%%%%%%%%%%%%%%%%%%%%%%%%%%%%%%%%%%%
\subsection{Multiplier Hopf Algebras\\(Purely Algebraic Setting, Van Daele)}
The main reference is \cite{Van-Daele2}.


%%%%%%%%%%%%%%%%%%%%%%%%%%%%%%%%%%%%%%%%%%%%%%%%%%%%%%%%%%%%%%%%%%%%%%%%%%%%%%%%%%%%%%%%%%%%%%%%%%%%%%%%%%%%%%%%%%%%%%%%%%
%%%%%%%%%%%%%%%%%%%%%%%%%%%%%%%%%%%%%%%%%%%%%%%%%%%%%%%%%%%%%%%%%%%%%%%%%%%%%%%%%%%%%%%%%%%%%%%%%%%%%%%%%%%%%%%%%%%%%%%%%%
\subsection{Algebraic Quantum Groups\\(Purely Algebraic Setting, Van Daele)}\label{2001301257}
An algebraic quantum group is a regular multiplier algebra with integrals. The main reference is \cite{Van-Daele1}.
See also the paper of Van Daele and Zhang in \cite{CaenepeelVan-Oystaeyen1}.
The category of algebraic quantum groups has a perfect duality theory. Also these objects has two subclasses discrete type
and compact type which are dual of each other. See the Subsections \ref{2001301300} and \ref{2001301301}

%%%%%%%%%%%%%%%%%%%%%%%%%%%%%%%%%%%%%%%%%%%%%%%%%%%%%%%%%%%%%%%%%%%%%%%%%%%%%%%%%%%%%%%%%%%%%%%%%%%%%%%%%%%%%%%%%%%%%%%%%%
%%%%%%%%%%%%%%%%%%%%%%%%%%%%%%%%%%%%%%%%%%%%%%%%%%%%%%%%%%%%%%%%%%%%%%%%%%%%%%%%%%%%%%%%%%%%%%%%%%%%%%%%%%%%%%%%%%%%%%%%%%
\subsection{Algebraic Quantum Groups of Discrete Type\\(Purely Algebraic Setting, Van Daele)}\label{2001301300}
The main reference is \cite{Van-Daele1}. An algebraic quantum group $A$ as in \ref{2001301257} is called of discrete type
if there is $h\in A$ with the property that $ah=\epsilon(a)h$ for every $a\in A$.
Any discrete quantum group as in \ref{2001281638} is an algebraic quantum group of discrete type but the
converse is not true in general.


%%%%%%%%%%%%%%%%%%%%%%%%%%%%%%%%%%%%%%%%%%%%%%%%%%%%%%%%%%%%%%%%%%%%%%%%%%%%%%%%%%%%%%%%%%%%%%%%%%%%%%%%%%%%%%%%%%%%%%%%%%
%%%%%%%%%%%%%%%%%%%%%%%%%%%%%%%%%%%%%%%%%%%%%%%%%%%%%%%%%%%%%%%%%%%%%%%%%%%%%%%%%%%%%%%%%%%%%%%%%%%%%%%%%%%%%%%%%%%%%%%%%%
\subsection{Algebraic Quantum Groups of Compact Type\\(Purely Algebraic Setting, Van Daele)}\label{2001301301}
The main reference is \cite{Van-Daele1}. An algebraic quantum group $A$ is called of type compact if $A$ is unital.
Any compact quantum group as in \ref{2001301307}. Any compact quantum group as in \ref{2001301307} has a dense *-subalgebra
which is an algebraic quantum group of compact type. See \cite[Theorem 1.2]{Woronowicz2} and also \cite{DijkhuizenKoornwinder1}.



%%%%%%%%%%%%%%%%%%%%%%%%%%%%%%%%%%%%%%%%%%%%%%%%%%%%%%%%%%%%%%%%%%%%%%%%%%%%%%%%%%%%%%%%%%%%%%%%%%%%%%%%%%%%%%%%%%%%%%%%%%
%%%%%%%%%%%%%%%%%%%%%%%%%%%%%%%%%%%%%%%%%%%%%%%%%%%%%%%%%%%%%%%%%%%%%%%%%%%%%%%%%%%%%%%%%%%%%%%%%%%%%%%%%%%%%%%%%%%%%%%%%%
\subsection{Weak Multiplier Hopf Algebras\\(Purely Algebraic Setting, Van Daele-Wang)}\label{2001301959}
The theory is developed in \cite{Van-DaeleWang1,Van-DaeleWang2,Van-DaeleWang3}.



%%%%%%%%%%%%%%%%%%%%%%%%%%%%%%%%%%%%%%%%%%%%%%%%%%%%%%%%%%%%%%%%%%%%%%%%%%%%%%%%%%%%%%%%%%%%%%%%%%%%%%%%%%%%%%%%%%%%%%%%%%
%%%%%%%%%%%%%%%%%%%%%%%%%%%%%%%%%%%%%%%%%%%%%%%%%%%%%%%%%%%%%%%%%%%%%%%%%%%%%%%%%%%%%%%%%%%%%%%%%%%%%%%%%%%%%%%%%%%%%%%%%%
\subsection{Algebraic Quantum Groupoid\\(Purely Algebraic Setting, Van Daele-Wang)}
It is a weak multiplier Hopf algebra as in \ref{2001301959} which is regular and has enough integrals.
(See the abstract of \cite{TimmermannVan-DaeleWang1} for the name.)
The theory is developed in \cite{Van-Daele9,Van-DaeleWang1,Van-DaeleWang2,Van-DaeleWang3}



%%%%%%%%%%%%%%%%%%%%%%%%%%%%%%%%%%%%%%%%%%%%%%%%%%%%%%%%%%%%%%%%%%%%%%%%%%%%%%%%%%%%%%%%%%%%%%%%%%%%%%%%%%%%%%%%%%%%%%%%%%
%%%%%%%%%%%%%%%%%%%%%%%%%%%%%%%%%%%%%%%%%%%%%%%%%%%%%%%%%%%%%%%%%%%%%%%%%%%%%%%%%%%%%%%%%%%%%%%%%%%%%%%%%%%%%%%%%%%%%%%%%%
\subsection{Regular Multiplier Hopf Algebroids\\(Purely Algebraic Setting, Timmermann-Van Daele)}\label{2001302033}
The main paper is \cite{TimmermannVan-Daele1}.
See \cite{TimmermannVan-Daele2} for relation between weak multiplier Hopf algebras and multiplier Hopf algebroids.


%%%%%%%%%%%%%%%%%%%%%%%%%%%%%%%%%%%%%%%%%%%%%%%%%%%%%%%%%%%%%%%%%%%%%%%%%%%%%%%%%%%%%%%%%%%%%%%%%%%%%%%%%%%%%%%%%%%%%%%%%%
%%%%%%%%%%%%%%%%%%%%%%%%%%%%%%%%%%%%%%%%%%%%%%%%%%%%%%%%%%%%%%%%%%%%%%%%%%%%%%%%%%%%%%%%%%%%%%%%%%%%%%%%%%%%%%%%%%%%%%%%%%
\subsection{Algebraic Quantum Groupoid\\(Purely Algebraic Setting, Timmermann)}
It is a regular multiplier Hopf algebroid as in \ref{2001302033} with a single faithful integral.
(See the abstract of \cite{TimmermannVan-DaeleWang1} for the name.)
The theory is developed in papers \cite{Timmermann1,Timmermann2}.

%%%%%%%%%%%%%%%%%%%%%%%%%%%%%%%%%%%%%%%%%%%%%%%%%%%%%%%%%%%%%%%%%%%%%%%%%%%%%%%%%%%%%%%%%%%%%%%%%%%%%%%%%%%%%%%%%%%%%%%%%%
%%%%%%%%%%%%%%%%%%%%%%%%%%%%%%%%%%%%%%%%%%%%%%%%%%%%%%%%%%%%%%%%%%%%%%%%%%%%%%%%%%%%%%%%%%%%%%%%%%%%%%%%%%%%%%%%%%%%%%%%%%
\subsection{Locally Compact Quantum Groupoid\\(C*-Algebraic Setting, Kahng-Van Daele)}
The theory is developed in papers \cite{KahngVan-Daele1,KahngVan-Daele2}.



%%%%%%%%%%%%%%%%%%%%%%%%%%%%%%%%%%%%%%%%%%%%%%%%%%%%%%%%%%%%%%%%%%%%%%%%%%%%%%%%%%%%%%%%%%%%%%%%%%%%%%%%%%%%%%%%%%%%%%%%%%
%%%%%%%%%%%%%%%%%%%%%%%%%%%%%%%%%%%%%%%%%%%%%%%%%%%%%%%%%%%%%%%%%%%%%%%%%%%%%%%%%%%%%%%%%%%%%%%%%%%%%%%%%%%%%%%%%%%%%%%%%%
\subsection{Duality}
One of the main references is \cite[Introductin]{EnockSchwartz1}.
The first paragraph of the introduction of \cite{MasudaNakagami1}:
``Classical Lie groups are important examples in the category of locally compact
groups. The general theory of unitary representations is developed for these objects as
harmonic analysis, which provides us a good theoretical framework for the detailed
study of the unitary representations of the classical Lie groups. This is regarded as an
extension of the Fourier analysis to a general context. For a locally compact group, its
dual i.e. the set of all the equivalence classes of irreducible unitary representations
plays an important role, and the duality established by Pontrjagin for Abelian groups,
Tannaka and Krein for compact groups, Steinspring for unimodular groups, Eymard
and Tatsuuma for locally compact groups is an important theoretical basis for the
harmonic analysis. On the other hand, at the formal level in the framework of pure
algebras, we use the notion of Hopf algebras to deal with the algebraic groups, discrete
groups, or the dual of those objects at the same time. Then functional analysis is
necessarily combined with the algebraic framework of Hopf algebras to have a good
control with the infinite dimensional unitary representations. This theory, especially
the argument utilized by Steinspring, suggests us to introduce the notion of Kac
algebras in the language of von Neumann algebras. The first take off from the group or
the group algebra to the Kac algebra was considered by Kac \cite{Kac1} and performed by
Takesaki \cite{Takesaki1} by introducing the, so-called, Kac-Takesaki operator or the fundamental
operator for the semifinite i.e. the unimodular case, and then completed by Enock and
Schwartz for the general case, in which the above mentioned duality was
established by Takesaki, Enock and Schwartz, and others \cite{StratilaVoiclescue1,VainermanKac1}.''

%%%%%%%%%%%%%%%%%%%%%%%%%%%%%%%%%%%%%%%%%%%%%%%%%%%%%%%%%%%%%%%%%%%%%%%%%%%%%%%%%%%%%%%%%%%%%%%%%%%%%%%%%%%%%%%%%%%%%%%%%%
%%%%%%%%%%%%%%%%%%%%%%%%%%%%%%%%%%%%%%%%%%%%%%%%%%%%%%%%%%%%%%%%%%%%%%%%%%%%%%%%%%%%%%%%%%%%%%%%%%%%%%%%%%%%%%%%%%%%%%%%%%
\subsection{Character and Super Character Theory}



%%%%%%%%%%%%%%%%%%%%%%%%%%%%%%%%%%%%%%%%%%%%%%%%%%%%%%%%%%%%%%%%%%%%%%%%%%%%%%%%%%%%%%%%%%%%%%%%%%%%%%%%%%%%%%%%%%%%%%%%%%
%%%%%%%%%%%%%%%%%%%%%%%%%%%%%%%%%%%%%%%%%%%%%%%%%%%%%%%%%%%%%%%%%%%%%%%%%%%%%%%%%%%%%%%%%%%%%%%%%%%%%%%%%%%%%%%%%%%%%%%%%%
\subsection{Fourier Transform}

%%%%%%%%%%%%%%%%%%%%%%%%%%%%%%%%%%%%%%%%%%%%%%%%%%%%%%%%%%%%%%%%%%%%%%%%%%%%%%%%%%%%%%%%%%%%%%%%%%%%%%%%%%%%%%%%%%%%%%%%%%
%%%%%%%%%%%%%%%%%%%%%%%%%%%%%%%%%%%%%%%%%%%%%%%%%%%%%%%%%%%%%%%%%%%%%%%%%%%%%%%%%%%%%%%%%%%%%%%%%%%%%%%%%%%%%%%%%%%%%%%%%%
\subsection{Amenability}




































































%%%%%%%%%%%%%%%%%%%%%%%%%%%%%%%%%%%%%%%%%%%%%%%%%%%%%%%%%%%%%%%%%%%%%%%%%%%%%%%%%%%%%%%%%%%%%%%%%%%%%%%%%%%%%%%%%%%%%%%%%%
%%%%%%%%%%%%%%%%%%%%%%%%%%%%%%%%%%%%%%%%%%%%%%%%%%%%%%%%%%%%%%%%%%%%%%%%%%%%%%%%%%%%%%%%%%%%%%%%%%%%%%%%%%%%%%%%%%%%%%%%%%
%%%%%%%%%%%%%%%%%%%%%%%%%%%%%%%%%%%%%%%%%%%%%%%%%%%%%%%%%%%%%%%%%%%%%%%%%%%%%%%%%%%%%%%%%%%%%%%%%%%%%%%%%%%%%%%%%%%%%%%%%%
%%%%%%%%%%%%%%%%%%%%%%%%%%%%%%%%%%%%%%%%%%%%%%%%%%%%%%%%%%%%%%%%%%%%%%%%%%%%%%%%%%%%%%%%%%%%%%%%%%%%%%%%%%%%%%%%%%%%%%%%%%
%%%%%%%%%%%%%%%%%%%%%%%%%%%%%%%%%%%%%%%%%%%%%%%%%%%%%%%%%%%%%%%%%%%%%%%%%%%%%%%%%%%%%%%%%%%%%%%%%%%%%%%%%%%%%%%%%%%%%%%%%%
%%%%%%%%%%%%%%%%%%%%%%%%%%%%%%%%%%%%%%%%%%%%%%%%%%%%%%%%%%%%%%%%%%%%%%%%%%%%%%%%%%%%%%%%%%%%%%%%%%%%%%%%%%%%%%%%%%%%%%%%%%
\section{Ph.D. Proposal: Some Aspects of Harmonic Analysis in Van Daele Category of Quantum Groups}
One of the extensively studied objects in mathematics is Hopf algebra.
Hopf algebras naturally appear in the abstract theory of rings and modules, in quantum algebra, in topological quantum field theory,
as the space of symmetries of quantum systems in mathematical physics, and in the mathematically rigorous theory of renormalization
of Feynman integrals. As it is well-known, in comparison with finite dimensional case, infinite dimensional
Hopf algebras are not so well-behaved objects from the purely algebraic viewpoint. In order to overcome this difficulty,
Van Daele introduced the notion of `multiplier Hopf algebra' in \cite{Van-Daele2}. It was recognized soon that the framework
of the theory of multiplier Hopf algebras is very relevant to study of various types of quantum groups.
Let $\b{D}$ denote a category whose objects are regular multiplier Hopf algebras with
invariant integrals, together with appropriate morphisms \cite{Van-Daele1}. We call $\b{D}$ Van Daele category
of quantum groups. The category $\b{D}$ has the following specificities:
\begin{enumerate}
\item[$\bullet$] $\b{D}$ contains multiplier Hopf algebras of discrete type \cite{Van-DaeleZhang2}.
\item[$\bullet$] Compact quantum groups \cite{Woronowicz1} can be considered as objects of $\b{D}$ of compact type.
\item[$\bullet$] The categories of compact and discrete groups can be canonically embedded in $\b{D}$.
\item[$\bullet$] Pontryagin duality between abelian compact and abelian discrete ordinary groups can be
generalized to a self-duality of $\b{D}$ \cite{Van-Daele1}.
\end{enumerate}
Van Dale and Wang in a series papers \cite{Van-DaeleWang1,Van-DaeleWang2,Van-DaeleWang3,Van-DaeleWang4}
have introduced the concept of weak multiplier Hopf-algebra. This concept can be considered as a Hopf algebraic analogue of the concept
of groupoid. We let $\b{D}_\r{w}$ denote the the category whose objects are regular weak multiplier Hopf algebras with
sufficiently many integrals. We call $\b{D}_\r{w}$ Van Daele category of quantum groupoids. $\b{D}$ can be considered as a full
subcategory of $\b{D}_\r{w}$. Also, it has been shown recently \cite{Van-DaeleWang4} that the self-duality of $\b{D}$
can be extended to a self-duality of $\b{D}_\r{w}$.

The aim of the thesis will be the study of some aspects of harmonic analysis in the Van Daele categories $\b{D}$ and $\b{D}_\r{w}$.
Specifically, we will consider the following problems.
\begin{problem}
\emph{A character on a (discrete) group is the composition of a finite dimensional representation of that group with the usual trace
on matrixes. Using irreducible characters on a group, one can associate an algebra to the group, called representation ring of the group, which
reflects all information encoded in the representation theory of the group. A supercharacter theory consists of a collection of
`approximately' irreducible characters satisfying certain axioms. The notion of `Supercharacter theory' originally has been defined
by Diaconis and Isaacs \cite{DiaconisIsaacs1} for finite groups. Recently, Keller in his thesis \cite{Keller1},
has generalized this theory and a related one called `theory of Schur rings' for an arbitrary finite dimensional Hopf-algebra. Also,
the theory of Schur rings has generalized recently for infinite groups in \cite{Bastian...1}.
One of the problems of the thesis, is to generalize supercharacter theory and theory of Schur rings for
at least objects of discrete type in $\b{D}$.
(Note that the thesis \cite{Keller1} has all relevant definitions and notions.
Also for character theory and representation rings see \cite{Isaacs1}. For a very good introduction to supercharacter theories and Schur rings see
\cite{Hendrickson1,Hendrickson2,Hendrickson3,Hendrickson4}.)}
\end{problem}
\begin{problem}
\emph{Consider the various aspects of harmonic analysis of groupoids (see \cite{Westman1} and its cites in Google Scholar)
for objects of $\b{D}_\r{w}$. An interesting and straightforward problem is to consider the analogue of Brandt groupoid
\cite{Sadr1,SadrPourabbas1} in $\b{D}_\r{w}$.}
\end{problem}
\begin{problem}
\emph{According to Bohr doctrine in quantum physics, all observable information of a quantum system in quantum laboratory,
are encoded in the classical parts of the mathematical models of the system.
Rigorously formulation of the Bohr doctrine, in these days, is called by mathematicians Bohrification of mathematical models of quantum systems
\cite{HeunenLandsmanSpitters1,Landsman1}. Recently, there are many attentions to Bohrification in noncommutative geometry,
see for example \cite{De-SilvaBarbosa1}. It seems that Bohrification will be a hot theme for research by mathematicians. So,
considering Bohrification for quantum groups must be a good problem. In the context of quantum group theory,
Bohrification, roughly speaking, means extracting the information
encoded in a multiplier Hopf algebra from its commutative subalgebras. I strongly suggest that Bohrification of quantum groups,
at least for objects of compact type in $\b{D}$, can be established very well.}
\end{problem}
\begin{problem}
\emph{Abstract pseudodifferential operators on compact groups and finite abelian groups have been
considered in \cite{MolahajlooPirhayati1,MolahajlooWong1}. It seems that, using Fourier theory in $\b{D}$ \cite{Van-Daele3},
pseudodifferential operators can be defined for objects of $\b{D}$ in the algebraic level, or for compact quantum groups
in the analytical level. Pseudodifferential operators on compact groups, recently is considered by Majid and his coworkers in
\cite{AkylzhanovMajidRuzhansky1}. But Majid's approach is seemingly very different from our approach. Comparing the two approach is also
a good problem.}
\end{problem}
%%%%%%%%%%%%%%%%%%%%%%%%%%%%%%%%%%%%%%%%%%%%%%%%%%%%%%%%%%%%%%%%%%%%%%%%%%%%%%%%%%%%%%%%%%%%%%%%%%%%%%%%%%%%%%%%%%%%%%%%%%
%%%%%%%%%%%%%%%%%%%%%%%%%%%%%%%%%%%%%%%%%%%%%%%%%%%%%%%%%%%%%%%%%%%%%%%%%%%%%%%%%%%%%%%%%%%%%%%%%%%%%%%%%%%%%%%%%%%%%%%%%%
%%%%%%%%%%%%%%%%%%%%%%%%%%%%%%%%%%%%%%%%%%%%%%%%%%%%%%%%%%%%%%%%%%%%%%%%%%%%%%%%%%%%%%%%%%%%%%%%%%%%%%%%%%%%%%%%%%%%%%%%%%
%%%%%%%%%%%%%%%%%%%%%%%%%%%%%%%%%%%%%%%%%%%%%%%%%%%%%%%%%%%%%%%%%%%%%%%%%%%%%%%%%%%%%%%%%%%%%%%%%%%%%%%%%%%%%%%%%%%%%%%%%%
%%%%%%%%%%%%%%%%%%%%%%%%%%%%%%%%%%%%%%%%%%%%%%%%%%%%%%%%%%%%%%%%%%%%%%%%%%%%%%%%%%%%%%%%%%%%%%%%%%%%%%%%%%%%%%%%%%%%%%%%%%
{\footnotesize\begin{thebibliography}{22}
%%%%%%%%%%%%%%%%%%%%%%%%%%%%%%%%%%%%%%%%%%%%%%%%%%%%%%%%%%%%%%%%%%%%%%%%%%%%%%%%%%%%%%%%%%%%%%%%%%%%%%%%%%%%%%%%%%%%%%%%%%
\bibitem{AkylzhanovMajidRuzhansky1}
R. Akylzhanov, S. Majid, M. Ruzhansky,
\emph{Smooth dense subalgebras and Fourier multipliers on compact quantum groups},
Communications in Mathematical Physics 362, no. 3 (2018): 761--799.
%%%%%%%%%%%%%%%%%%%%%%%%%%%%%%%%%%%%%%%%%%%%%%%%%%%%%%%%%%%%%%%%%%%%%%%%%%%%%%%%%%%%%%%%%%%%%%%%%%%%%%%%%%%%%%%%%%%%%%%%%%
\bibitem{Bastian...1}
N. Bastian, J. Brewer, S. Humphries, A. Misseldine, C. Thompson,
\emph{On Schur rings over infinite groups},
Algebras and Representation Theory (2018): 1--19.
%%%%%%%%%%%%%%%%%%%%%%%%%%%%%%%%%%%%%%%%%%%%%%%%%%%%%%%%%%%%%%%%%%%%%%%%%%%%%%%%%%%%%%%%%%%%%%%%%%%%%%%%%%%%%%%%%%%%%%%%%%
\bibitem{CaenepeelVan-Oystaeyen1}
S. Caenepeel, F Van Oystaeyen,
\emph{Hopf algebras and quantum groups},
CRC Press, 2000.
%%%%%%%%%%%%%%%%%%%%%%%%%%%%%%%%%%%%%%%%%%%%%%%%%%%%%%%%%%%%%%%%%%%%%%%%%%%%%%%%%%%%%%%%%%%%%%%%%%%%%%%%%%%%%%%%%%%%%%%%%%
\bibitem{De-SilvaBarbosa1}
N. De Silva, R.S. Barbosa,
\emph{Contextuality and noncommutative geometry in quantum mechanics},
arXiv:1806.02840 (2018).
%%%%%%%%%%%%%%%%%%%%%%%%%%%%%%%%%%%%%%%%%%%%%%%%%%%%%%%%%%%%%%%%%%%%%%%%%%%%%%%%%%%%%%%%%%%%%%%%%%%%%%%%%%%%%%%%%%%%%%%%%%
\bibitem{DeitmarEchterhoff1}
A. Deitmar, S. Echterhoff,
\emph{Principles of harmonic analysis},
Springer, 2014.
%%%%%%%%%%%%%%%%%%%%%%%%%%%%%%%%%%%%%%%%%%%%%%%%%%%%%%%%%%%%%%%%%%%%%%%%%%%%%%%%%%%%%%%%%%%%%%%%%%%%%%%%%%%%%%%%%%%%%%%%%%
\bibitem{DelvauxVan-Daele1}
L. Delvaux, A. Van Daele,
\emph{Algebraic quantum hypergroups}
Advances in Mathematics 226, no. 2 (2011): 1134--1167.
%%%%%%%%%%%%%%%%%%%%%%%%%%%%%%%%%%%%%%%%%%%%%%%%%%%%%%%%%%%%%%%%%%%%%%%%%%%%%%%%%%%%%%%%%%%%%%%%%%%%%%%%%%%%%%%%%%%%%%%%%%
\bibitem{DiaconisIsaacs1}
P. Diaconis, I. Isaacs,
\emph{Supercharacters and superclasses for algebra groups},
Transactions of the American Mathematical Society 360, no. 5 (2008): 2359--2392.
%%%%%%%%%%%%%%%%%%%%%%%%%%%%%%%%%%%%%%%%%%%%%%%%%%%%%%%%%%%%%%%%%%%%%%%%%%%%%%%%%%%%%%%%%%%%%%%%%%%%%%%%%%%%%%%%%%%%%%%%%%
\bibitem{DijkhuizenKoornwinder1}
M. Dijkhuizen, T. Koornwinder,
\emph{CQG algebras: A direct algebraic approach to compact quantum groups}
Lett. Math. Phys. 32 (1994), 315--330.
%%%%%%%%%%%%%%%%%%%%%%%%%%%%%%%%%%%%%%%%%%%%%%%%%%%%%%%%%%%%%%%%%%%%%%%%%%%%%%%%%%%%%%%%%%%%%%%%%%%%%%%%%%%%%%%%%%%%%%%%%%
\bibitem{EnockSchwartz1}
M. Enock, J.-M. Schwartz,
\emph{Kac algebras and duality of locally compact groups},
Springer Science \& Business Media, 2013.
%%%%%%%%%%%%%%%%%%%%%%%%%%%%%%%%%%%%%%%%%%%%%%%%%%%%%%%%%%%%%%%%%%%%%%%%%%%%%%%%%%%%%%%%%%%%%%%%%%%%%%%%%%%%%%%%%%%%%%%%%%
\bibitem{Hendrickson1}
A.O.F. Hendrickson,
\emph{Supercharacter theories, research statement},
faculty.cord.edu/ahendric.
%%%%%%%%%%%%%%%%%%%%%%%%%%%%%%%%%%%%%%%%%%%%%%%%%%%%%%%%%%%%%%%%%%%%%%%%%%%%%%%%%%%%%%%%%%%%%%%%%%%%%%%%%%%%%%%%%%%%%%%%%%
\bibitem{Hendrickson2}
A.O.F. Hendrickson,
\emph{Construction of supercharacter theories of finite groups},
arXiv:0905.3538 (2009).
%%%%%%%%%%%%%%%%%%%%%%%%%%%%%%%%%%%%%%%%%%%%%%%%%%%%%%%%%%%%%%%%%%%%%%%%%%%%%%%%%%%%%%%%%%%%%%%%%%%%%%%%%%%%%%%%%%%%%%%%%%
\bibitem{Hendrickson3}
A.O.F. Hendrickson,
\emph{Supercharacter theories and Schur rings},
arXiv:1006.1363 (2010).
%%%%%%%%%%%%%%%%%%%%%%%%%%%%%%%%%%%%%%%%%%%%%%%%%%%%%%%%%%%%%%%%%%%%%%%%%%%%%%%%%%%%%%%%%%%%%%%%%%%%%%%%%%%%%%%%%%%%%%%%%%
\bibitem{Hendrickson4}
A.O.F. Hendrickson,
\emph{Supercharacter theory constructions corresponding to Schur ring products},
Communications in Algebra 40, no. 12 (2012): 4420--4438.
%%%%%%%%%%%%%%%%%%%%%%%%%%%%%%%%%%%%%%%%%%%%%%%%%%%%%%%%%%%%%%%%%%%%%%%%%%%%%%%%%%%%%%%%%%%%%%%%%%%%%%%%%%%%%%%%%%%%%%%%%%
\bibitem{HeunenLandsmanSpitters1}
C. Heunen, N.P. Landsman, B. Spitters,
\emph{Bohrification},
arXiv:0909.3468 (2009).
%%%%%%%%%%%%%%%%%%%%%%%%%%%%%%%%%%%%%%%%%%%%%%%%%%%%%%%%%%%%%%%%%%%%%%%%%%%%%%%%%%%%%%%%%%%%%%%%%%%%%%%%%%%%%%%%%%%%%%%%%%
\bibitem{IllanesNadler1}
A. Illanes, S. Nadler,
\emph{Hyperspaces: fundamentals and recent advances},
Vol. 216 CRC Press, 1999.
%%%%%%%%%%%%%%%%%%%%%%%%%%%%%%%%%%%%%%%%%%%%%%%%%%%%%%%%%%%%%%%%%%%%%%%%%%%%%%%%%%%%%%%%%%%%%%%%%%%%%%%%%%%%%%%%%%%%%%%%%%
\bibitem{Isaacs1}
I.M. Isaacs,
\emph{Character theory of finite groups},
Vol. 69. Courier Corporation, 1994.
%%%%%%%%%%%%%%%%%%%%%%%%%%%%%%%%%%%%%%%%%%%%%%%%%%%%%%%%%%%%%%%%%%%%%%%%%%%%%%%%%%%%%%%%%%%%%%%%%%%%%%%%%%%%%%%%%%%%%%%%%%
\bibitem{Kac1}
G.I. Kac,
\emph{Generalization of the group principle of duality}
Soviet Math, Dokl., 2(1961), 581--584.
%%%%%%%%%%%%%%%%%%%%%%%%%%%%%%%%%%%%%%%%%%%%%%%%%%%%%%%%%%%%%%%%%%%%%%%%%%%%%%%%%%%%%%%%%%%%%%%%%%%%%%%%%%%%%%%%%%%%%%%%%%
\bibitem{KahngVan-Daele1}
B.-J. Kahng, A. Van Daele,
\emph{A class of C*-algebraic locally compact quantum groupoids part I. Motivation and definition},
International Journal of Mathematics 29, no. 04 (2018): 1850029.
%%%%%%%%%%%%%%%%%%%%%%%%%%%%%%%%%%%%%%%%%%%%%%%%%%%%%%%%%%%%%%%%%%%%%%%%%%%%%%%%%%%%%%%%%%%%%%%%%%%%%%%%%%%%%%%%%%%%%%%%%%
\bibitem{KahngVan-Daele2}
B.-J. Kahng, A. Van Daele,
\emph{A class of C*-algebraic locally compact quantum groupoids Part II, Main theory},
Advances in Mathematics 354 (2019): 106761.
%%%%%%%%%%%%%%%%%%%%%%%%%%%%%%%%%%%%%%%%%%%%%%%%%%%%%%%%%%%%%%%%%%%%%%%%%%%%%%%%%%%%%%%%%%%%%%%%%%%%%%%%%%%%%%%%%%%%%%%%%%
\bibitem{Keller1}
J.C. Keller,
\emph{Generalized supercharacter theories and Schur rings for Hopf algebras},
Ph.D. Thesis, University of Colorado (2014).
%%%%%%%%%%%%%%%%%%%%%%%%%%%%%%%%%%%%%%%%%%%%%%%%%%%%%%%%%%%%%%%%%%%%%%%%%%%%%%%%%%%%%%%%%%%%%%%%%%%%%%%%%%%%%%%%%%%%%%%%%%
\bibitem{Kustermans1}
J. Kustermans
\emph{Locally compact quantum groups in the universal setting},
International Journal of Mathematics 12, no. 03 (2001): 289--338.
%%%%%%%%%%%%%%%%%%%%%%%%%%%%%%%%%%%%%%%%%%%%%%%%%%%%%%%%%%%%%%%%%%%%%%%%%%%%%%%%%%%%%%%%%%%%%%%%%%%%%%%%%%%%%%%%%%%%%%%%%%
\bibitem{KustermansVaes2}
J. Kustermans, S. Vaes
\emph{Locally compact quantum groups in the von Neumann algebraic setting},
Mathematica Scandinavica (2003): 68--92.
%%%%%%%%%%%%%%%%%%%%%%%%%%%%%%%%%%%%%%%%%%%%%%%%%%%%%%%%%%%%%%%%%%%%%%%%%%%%%%%%%%%%%%%%%%%%%%%%%%%%%%%%%%%%%%%%%%%%%%%%%%
\bibitem{KustermansVaes1}
J. Kustermans, S. Vaes
\emph{Locally compact quantum groups},
Annales Scientifiques de l�Ecole Normale Sup�rieure, vol. 33, no. 6, pp. 837--934.
%%%%%%%%%%%%%%%%%%%%%%%%%%%%%%%%%%%%%%%%%%%%%%%%%%%%%%%%%%%%%%%%%%%%%%%%%%%%%%%%%%%%%%%%%%%%%%%%%%%%%%%%%%%%%%%%%%%%%%%%%%
\bibitem{Landsman1}
K. Landsman,
\emph{Bohrification: from classical concepts to commutative algebras},
arXiv:1601.02794 (2016).
%%%%%%%%%%%%%%%%%%%%%%%%%%%%%%%%%%%%%%%%%%%%%%%%%%%%%%%%%%%%%%%%%%%%%%%%%%%%%%%%%%%%%%%%%%%%%%%%%%%%%%%%%%%%%%%%%%%%%%%%%%
\bibitem{Lledo1}
F. Lledo,
\emph{Modular Theory by example},
(2009).
%%%%%%%%%%%%%%%%%%%%%%%%%%%%%%%%%%%%%%%%%%%%%%%%%%%%%%%%%%%%%%%%%%%%%%%%%%%%%%%%%%%%%%%%%%%%%%%%%%%%%%%%%%%%%%%%%%%%%%%%%%
\bibitem{MaesVan-Daele1}
A. Maes, A. Van Daele,
\emph{Notes on compact quantum groups}
(1998).
%%%%%%%%%%%%%%%%%%%%%%%%%%%%%%%%%%%%%%%%%%%%%%%%%%%%%%%%%%%%%%%%%%%%%%%%%%%%%%%%%%%%%%%%%%%%%%%%%%%%%%%%%%%%%%%%%%%%%%%%%%
\bibitem{MasudaNakagami1}
T. Masuda, Y. Nakagami,
\emph{A von Neumann algebra framework for the duality of the quantum groupss},
Publ. RIMS Kyoto Univ., 30(1994), 799--850.
%%%%%%%%%%%%%%%%%%%%%%%%%%%%%%%%%%%%%%%%%%%%%%%%%%%%%%%%%%%%%%%%%%%%%%%%%%%%%%%%%%%%%%%%%%%%%%%%%%%%%%%%%%%%%%%%%%%%%%%%%%
\bibitem{MasudaNakagamiWoronowicz1}
T. Masuda, Y. Nakagami, S.L. Woronowicz,
\emph{A C*-algebraic framework for quantum groups},
International Journal of Mathematics 14, no. 09 (2003): 903--1001.
%%%%%%%%%%%%%%%%%%%%%%%%%%%%%%%%%%%%%%%%%%%%%%%%%%%%%%%%%%%%%%%%%%%%%%%%%%%%%%%%%%%%%%%%%%%%%%%%%%%%%%%%%%%%%%%%%%%%%%%%%%
\bibitem{MolahajlooPirhayati1}
S. Molahajloo, M. Pirhayati,
\emph{Traces of pseudo-differential operators on compact and Hausdorff groups},
Journal of Pseudo-Differential Operators and Applications 4, no. 3 (2013): 361--369
%%%%%%%%%%%%%%%%%%%%%%%%%%%%%%%%%%%%%%%%%%%%%%%%%%%%%%%%%%%%%%%%%%%%%%%%%%%%%%%%%%%%%%%%%%%%%%%%%%%%%%%%%%%%%%%%%%%%%%%%%%
\bibitem{MolahajlooWong1}
S. Molahajloo, K.L. Wong,
\emph{Pseudo-differential operators on finite abelian groups},
Journal of Pseudo-Differential Operators and Applications 6, no. 1 (2015): 1--9.
%%%%%%%%%%%%%%%%%%%%%%%%%%%%%%%%%%%%%%%%%%%%%%%%%%%%%%%%%%%%%%%%%%%%%%%%%%%%%%%%%%%%%%%%%%%%%%%%%%%%%%%%%%%%%%%%%%%%%%%%%%
\bibitem{NeshveyevTuset1}
S. Neshveyev, L. Tuset,
\emph{Compact quantum groups and their representation categories},
Vol. 20 Paris: Societe mathematique de France, 2013.
%%%%%%%%%%%%%%%%%%%%%%%%%%%%%%%%%%%%%%%%%%%%%%%%%%%%%%%%%%%%%%%%%%%%%%%%%%%%%%%%%%%%%%%%%%%%%%%%%%%%%%%%%%%%%%%%%%%%%%%%%%
\bibitem{Sadr1}
M.M. Sadr,
\emph{Pseudo-amenability of Brandt semigroup algebras},
Comm. Math. Univ. Carolinae, vol. 50 (2009), 413--419.
%%%%%%%%%%%%%%%%%%%%%%%%%%%%%%%%%%%%%%%%%%%%%%%%%%%%%%%%%%%%%%%%%%%%%%%%%%%%%%%%%%%%%%%%%%%%%%%%%%%%%%%%%%%%%%%%%%%%%%%%%%
\bibitem{SadrPourabbas1}
M.M. Sadr, A. Pourabbas,
\emph{Approximate amenability of Banach category algebras with application to semigroup algebras},
Semigroup Forum, 79 (2009), 55--64.
%%%%%%%%%%%%%%%%%%%%%%%%%%%%%%%%%%%%%%%%%%%%%%%%%%%%%%%%%%%%%%%%%%%%%%%%%%%%%%%%%%%%%%%%%%%%%%%%%%%%%%%%%%%%%%%%%%%%%%%%%%
\bibitem{StratilaVoiclescue1}
S. Stratila, D. Voiclescue,
\emph{On the crossed products, I, II}
Rev. Roum. Math. Pures Appl., 21 (1976), 1411--1449; 22 (1977), 83--117.
%%%%%%%%%%%%%%%%%%%%%%%%%%%%%%%%%%%%%%%%%%%%%%%%%%%%%%%%%%%%%%%%%%%%%%%%%%%%%%%%%%%%%%%%%%%%%%%%%%%%%%%%%%%%%%%%%%%%%%%%%%
\bibitem{Takesaki1}
M. Takesaki,
\emph{Duality and von Neumann algebras},
Lecture Notes in Math., 247(1972), 665--779,
Springer-Verlag, Berlin-Heiderberg-New York.
%%%%%%%%%%%%%%%%%%%%%%%%%%%%%%%%%%%%%%%%%%%%%%%%%%%%%%%%%%%%%%%%%%%%%%%%%%%%%%%%%%%%%%%%%%%%%%%%%%%%%%%%%%%%%%%%%%%%%%%%%%
\bibitem{Timmermann1}
T. Timmermann,
\emph{Integration on algebraic quantum groupoids},
Int. J. Math. 27 (2016).
%%%%%%%%%%%%%%%%%%%%%%%%%%%%%%%%%%%%%%%%%%%%%%%%%%%%%%%%%%%%%%%%%%%%%%%%%%%%%%%%%%%%%%%%%%%%%%%%%%%%%%%%%%%%%%%%%%%%%%%%%%
\bibitem{Timmermann2}
T. Timmermann,
\emph{On duality of algebraic quantum groupoids},
Adv. in Math. 309 (2017) 692�-746.
%%%%%%%%%%%%%%%%%%%%%%%%%%%%%%%%%%%%%%%%%%%%%%%%%%%%%%%%%%%%%%%%%%%%%%%%%%%%%%%%%%%%%%%%%%%%%%%%%%%%%%%%%%%%%%%%%%%%%%%%%%
\bibitem{TimmermannVan-Daele1}
T. Timmermann, A. Van Daele,
\emph{Regular multiplier Hopf algebroids, Basic theory and examples},
Commun. Alg. 46 (2017), 1926--1958.
%%%%%%%%%%%%%%%%%%%%%%%%%%%%%%%%%%%%%%%%%%%%%%%%%%%%%%%%%%%%%%%%%%%%%%%%%%%%%%%%%%%%%%%%%%%%%%%%%%%%%%%%%%%%%%%%%%%%%%%%%%
\bibitem{TimmermannVan-Daele2}
T. Timmermann, A. Van Daele,
\emph{Multiplier Hopf algebroids arising from weak multiplier Hopf algebras},
Banach Center Publications, 106 (2015), 73-�110.
%%%%%%%%%%%%%%%%%%%%%%%%%%%%%%%%%%%%%%%%%%%%%%%%%%%%%%%%%%%%%%%%%%%%%%%%%%%%%%%%%%%%%%%%%%%%%%%%%%%%%%%%%%%%%%%%%%%%%%%%%%
\bibitem{TimmermannVan-DaeleWang1}
T. Timmermann, A. Van Daele, S. Wang,
\emph{Pairing and duality of algebraic quantum groupoids},
(2019).
%%%%%%%%%%%%%%%%%%%%%%%%%%%%%%%%%%%%%%%%%%%%%%%%%%%%%%%%%%%%%%%%%%%%%%%%%%%%%%%%%%%%%%%%%%%%%%%%%%%%%%%%%%%%%%%%%%%%%%%%%%
\bibitem{Uijlen1}
S. Uijlen,
\emph{Categorical Aspects of von Neumann Algebras and AW*-algebras}
2013.
%%%%%%%%%%%%%%%%%%%%%%%%%%%%%%%%%%%%%%%%%%%%%%%%%%%%%%%%%%%%%%%%%%%%%%%%%%%%%%%%%%%%%%%%%%%%%%%%%%%%%%%%%%%%%%%%%%%%%%%%%%
\bibitem{VainermanKac1}
L.I. Vainerman, G.I. Kac,
\emph{Non unimodular ring groups and Hopf-von Neumann algebras},
Math, USSR Sbornik, 23 (1974), 185--214.
%%%%%%%%%%%%%%%%%%%%%%%%%%%%%%%%%%%%%%%%%%%%%%%%%%%%%%%%%%%%%%%%%%%%%%%%%%%%%%%%%%%%%%%%%%%%%%%%%%%%%%%%%%%%%%%%%%%%%%%%%%
\bibitem{Van-Daele2}
A. Van Daele,
\emph{Multiplier Hopf algebras},
Trans. Amer. Math. Soc. 342 (1994), 917--932.
%%%%%%%%%%%%%%%%%%%%%%%%%%%%%%%%%%%%%%%%%%%%%%%%%%%%%%%%%%%%%%%%%%%%%%%%%%%%%%%%%%%%%%%%%%%%%%%%%%%%%%%%%%%%%%%%%%%%%%%%%%
\bibitem{Van-Daele1}
A. Van Daele,
\emph{An algebraic framework for group duality},
Advances in Mathematics 140, no. 2 (1998): 323--366.
%%%%%%%%%%%%%%%%%%%%%%%%%%%%%%%%%%%%%%%%%%%%%%%%%%%%%%%%%%%%%%%%%%%%%%%%%%%%%%%%%%%%%%%%%%%%%%%%%%%%%%%%%%%%%%%%%%%%%%%%%%
\bibitem{Van-Daele3}
A. Van Daele,
\emph{The Fourier transform in quantum group theory},
arXiv preprint math/0609502 (2006).
%%%%%%%%%%%%%%%%%%%%%%%%%%%%%%%%%%%%%%%%%%%%%%%%%%%%%%%%%%%%%%%%%%%%%%%%%%%%%%%%%%%%%%%%%%%%%%%%%%%%%%%%%%%%%%%%%%%%%%%%%%
\bibitem{Van-Daele4}
A. Van Daele,
\emph{Locally compact quantum groups: the von Neumann algebra versus the C*-algebra approach}
Bull. Kerala Math. Assoc. (2005), 153--177.
%%%%%%%%%%%%%%%%%%%%%%%%%%%%%%%%%%%%%%%%%%%%%%%%%%%%%%%%%%%%%%%%%%%%%%%%%%%%%%%%%%%%%%%%%%%%%%%%%%%%%%%%%%%%%%%%%%%%%%%%%%
\bibitem{Van-Daele5}
A. Van Daele,
\emph{Locally compact quantum groups: A von Neumann algebra approach},
(2006).
%%%%%%%%%%%%%%%%%%%%%%%%%%%%%%%%%%%%%%%%%%%%%%%%%%%%%%%%%%%%%%%%%%%%%%%%%%%%%%%%%%%%%%%%%%%%%%%%%%%%%%%%%%%%%%%%%%%%%%%%%%
\bibitem{Van-Daele6}
A. Van Daele,
\emph{Discrete Quantum Groups},
Discrete Quantum Groups, Journal of Algebra 180 (1996), 431--444.
%%%%%%%%%%%%%%%%%%%%%%%%%%%%%%%%%%%%%%%%%%%%%%%%%%%%%%%%%%%%%%%%%%%%%%%%%%%%%%%%%%%%%%%%%%%%%%%%%%%%%%%%%%%%%%%%%%%%%%%%%%
\bibitem{Van-Daele7}
A. Van Daele,
\emph{Tools for working with multiplier Hopf algebras},
(2008)
%%%%%%%%%%%%%%%%%%%%%%%%%%%%%%%%%%%%%%%%%%%%%%%%%%%%%%%%%%%%%%%%%%%%%%%%%%%%%%%%%%%%%%%%%%%%%%%%%%%%%%%%%%%%%%%%%%%%%%%%%%
\bibitem{Van-Daele8}
Van Daele,
\emph{From Hopf algebras to topological quantum groups A short history, various aspects and some problems},
(2019).
%%%%%%%%%%%%%%%%%%%%%%%%%%%%%%%%%%%%%%%%%%%%%%%%%%%%%%%%%%%%%%%%%%%%%%%%%%%%%%%%%%%%%%%%%%%%%%%%%%%%%%%%%%%%%%%%%%%%%%%%%%
\bibitem{Van-Daele9}
Van Daele,
\emph{Algebraic quantum groupoids, An example},
(2017).
%%%%%%%%%%%%%%%%%%%%%%%%%%%%%%%%%%%%%%%%%%%%%%%%%%%%%%%%%%%%%%%%%%%%%%%%%%%%%%%%%%%%%%%%%%%%%%%%%%%%%%%%%%%%%%%%%%%%%%%%%%
\bibitem{Van-DaeleWang1}
A. Van Daele, S. Wang,
\emph{Weak multiplier Hopf algebras, preliminaries, motivation and basic examples},
arXiv:1210.3954 (2012).
%%%%%%%%%%%%%%%%%%%%%%%%%%%%%%%%%%%%%%%%%%%%%%%%%%%%%%%%%%%%%%%%%%%%%%%%%%%%%%%%%%%%%%%%%%%%%%%%%%%%%%%%%%%%%%%%%%%%%%%%%%
\bibitem{Van-DaeleWang2}
A. Van Daele, S. Wang,
\emph{Weak multiplier Hopf algebras I, the main theory},
Journal f�r Die Reine und Angewandte Mathematik (Crelles Journal), 2015(705), pp.155--209.
%%%%%%%%%%%%%%%%%%%%%%%%%%%%%%%%%%%%%%%%%%%%%%%%%%%%%%%%%%%%%%%%%%%%%%%%%%%%%%%%%%%%%%%%%%%%%%%%%%%%%%%%%%%%%%%%%%%%%%%%%%
\bibitem{Van-DaeleWang3}
A. Van Daele, S. Wang,
\emph{Weak multiplier Hopf algebras II, the source and target algebras},
arXiv:1403.7906 (2014).
%%%%%%%%%%%%%%%%%%%%%%%%%%%%%%%%%%%%%%%%%%%%%%%%%%%%%%%%%%%%%%%%%%%%%%%%%%%%%%%%%%%%%%%%%%%%%%%%%%%%%%%%%%%%%%%%%%%%%%%%%%
\bibitem{Van-DaeleWang4}
A. Van Daele, S. Wang,
\emph{Weak multiplier Hopf algebras III, Integrals and Duality},
arXiv:1701.04951 (2017).
%%%%%%%%%%%%%%%%%%%%%%%%%%%%%%%%%%%%%%%%%%%%%%%%%%%%%%%%%%%%%%%%%%%%%%%%%%%%%%%%%%%%%%%%%%%%%%%%%%%%%%%%%%%%%%%%%%%%%%%%%%
\bibitem{Van-DaeleZhang2}
A. Van Daele, Y. Zhang,
\emph{Multiplier Hopf algebras of discrete type},
Journal of Algebra 214, no. 2 (1999): 400-417.
%%%%%%%%%%%%%%%%%%%%%%%%%%%%%%%%%%%%%%%%%%%%%%%%%%%%%%%%%%%%%%%%%%%%%%%%%%%%%%%%%%%%%%%%%%%%%%%%%%%%%%%%%%%%%%%%%%%%%%%%%%
\bibitem{Westman1}
J. Westman,
\emph{Harmonic analysis on groupoids},
Pacific Journal of Mathematics 27, no. 3 (1968): 621--632.
%%%%%%%%%%%%%%%%%%%%%%%%%%%%%%%%%%%%%%%%%%%%%%%%%%%%%%%%%%%%%%%%%%%%%%%%%%%%%%%%%%%%%%%%%%%%%%%%%%%%%%%%%%%%%%%%%%%%%%%%%%
\bibitem{Woronowicz1}
S.L. Woronowicz,
\emph{Compact quantum groups},
Sym\'{e}tries Quantiques (Les Houches, 1995) 845, no. 884 (1998): 98.
%%%%%%%%%%%%%%%%%%%%%%%%%%%%%%%%%%%%%%%%%%%%%%%%%%%%%%%%%%%%%%%%%%%%%%%%%%%%%%%%%%%%%%%%%%%%%%%%%%%%%%%%%%%%%%%%%%%%%%%%%%
\bibitem{Woronowicz2}
S.L. Woronowicz,
\emph{Compact matrix pseudogroups},
Communications in Mathematical Physics 111, no. 4 (1987): 613--665.
%%%%%%%%%%%%%%%%%%%%%%%%%%%%%%%%%%%%%%%%%%%%%%%%%%%%%%%%%%%%%%%%%%%%%%%%%%%%%%%%%%%%%%%%%%%%%%%%%%%%%%%%%%%%%%%%%%%%%%%%%%
%%%%%%%%%%%%%%%%%%%%%%%%%%%%%%%%%%%%%%%%%%%%%%%%%%%%%%%%%%%%%%%%%%%%%%%%%%%%%%%%%%%%%%%%%%%%%%%%%%%%%%%%%%%%%%%%%%%%%%%%%%
%%%%%%%%%%%%%%%%%%%%%%%%%%%%%%%%%%%%%%%%%%%%%%%%%%%%%%%%%%%%%%%%%%%%%%%%%%%%%%%%%%%%%%%%%%%%%%%%%%%%%%%%%%%%%%%%%%%%%%%%%%
\end{thebibliography}}
%%%%%%%%%%%%%%%%%%%%%%%%%%%%%%%%%%%%%%%%%%%%%%%%%%%%%%%%%%%%%%%%%%%%%%%%%%%%%%%%%%%%%%%%%%%%%%%%%%%%%%%%%%%%%%%%%%%%%%%%%%
%%%%%%%%%%%%%%%%%%%%%%%%%%%%%%%%%%%%%%%%%%%%%%%%%%%%%%%%%%%%%%%%%%%%%%%%%%%%%%%%%%%%%%%%%%%%%%%%%%%%%%%%%%%%%%%%%%%%%%%%%%
%%%%%%%%%%%%%%%%%%%%%%%%%%%%%%%%%%%%%%%%%%%%%%%%%%%%%%%%%%%%%%%%%%%%%%%%%%%%%%%%%%%%%%%%%%%%%%%%%%%%%%%%%%%%%%%%%%%%%%%%%%
%%%%%%%%%%%%%%%%%%%%%%%%%%%%%%%%%%%%%%%%%%%%%%%%%%%%%%%%%%%%%%%%%%%%%%%%%%%%%%%%%%%%%%%%%%%%%%%%%%%%%%%%%%%%%%%%%%%%%%%%%%
%%%%%%%%%%%%%%%%%%%%%%%%%%%%%%%%%%%%%%%%%%%%%%%%%%%%%%%%%%%%%%%%%%%%%%%%%%%%%%%%%%%%%%%%%%%%%%%%%%%%%%%%%%%%%%%%%%%%%%%%%%
%%%%%%%%%%%%%%%%%%%%%%%%%%%%%%%%%%%%%%%%%%%%%%%%%%%%%%%%%%%%%%%%%%%%%%%%%%%%%%%%%%%%%%%%%%%%%%%%%%%%%%%%%%%%%%%%%%%%%%%%%%
\end{document}





%%%%%%%%%%%%%%%%%%%%%%%%%%%%%%%%%%%%%%%%%%%%%%%%%%%%%%%%%%%%%%%%%%%%%%%%%%%%%%%%%%%%%%%%%%%%%%%%%%%%%%%%%%%%%%%%%%%%%%%%%%
\bibitem{Uijlen}
S. Uijlen,
\emph{Categorical Aspects of von Neumann Algebras and AW*-algebras}
2013.
Uijlen Categorical Aspects of von Neumann Algebras and AW-star-algebras





%%%%%%%%%%%%%%%%%%%%%%%%%%%%%%%%%%%%%%%%%%%%%%%%%%%%%%%%%%%%%%%%%%%%%%%%%%%%%%%%%%%%%%%%%%%%%%%%%%%%%%%%%%%%%%%%%%%%%%%%%%
\bibitem{}
\emph{}







%%%%%%%%%%%%%%%%%%%%%%%%%%%%%%%%%%%%%%%%%%%%%%%%%%%%%%%%%%%%%%%%%%%%%%%%%%%%%%%%%%%%%%%%%%%%%%%%%%%%%%%%%%%%%%%%%%%%%%%%%%
\bibitem{}
\emph{}







%%%%%%%%%%%%%%%%%%%%%%%%%%%%%%%%%%%%%%%%%%%%%%%%%%%%%%%%%%%%%%%%%%%%%%%%%%%%%%%%%%%%%%%%%%%%%%%%%%%%%%%%%%%%%%%%%%%%%%%%%%
\bibitem{}
\emph{}








%%%%%%%%%%%%%%%%%%%%%%%%%%%%%%%%%%%%%%%%%%%%%%%%%%%%%%%%%%%%%%%%%%%%%%%%%%%%%%%%%%%%%%%%%%%%%%%%%%%%%%%%%%%%%%%%%%%%%%%%%%
\bibitem{}
\emph{}







%%%%%%%%%%%%%%%%%%%%%%%%%%%%%%%%%%%%%%%%%%%%%%%%%%%%%%%%%%%%%%%%%%%%%%%%%%%%%%%%%%%%%%%%%%%%%%%%%%%%%%%%%%%%%%%%%%%%%%%%%%
\bibitem{}
\emph{}







%%%%%%%%%%%%%%%%%%%%%%%%%%%%%%%%%%%%%%%%%%%%%%%%%%%%%%%%%%%%%%%%%%%%%%%%%%%%%%%%%%%%%%%%%%%%%%%%%%%%%%%%%%%%%%%%%%%%%%%%%%
\bibitem{}
\emph{}







%%%%%%%%%%%%%%%%%%%%%%%%%%%%%%%%%%%%%%%%%%%%%%%%%%%%%%%%%%%%%%%%%%%%%%%%%%%%%%%%%%%%%%%%%%%%%%%%%%%%%%%%%%%%%%%%%%%%%%%%%%
\bibitem{}
\emph{}







%%%%%%%%%%%%%%%%%%%%%%%%%%%%%%%%%%%%%%%%%%%%%%%%%%%%%%%%%%%%%%%%%%%%%%%%%%%%%%%%%%%%%%%%%%%%%%%%%%%%%%%%%%%%%%%%%%%%%%%%%%
\bibitem{}
\emph{}





















